\documentclass[./../main.tex]{subfiles}
\graphicspath{{img/}}

\begin{document}
\begin{problema}
	Calcula los eigenvalores y lo eigenvectores de la matrix \(X\) de Pauli.

	\begin{equation*}
		\op{X} = \begin{pNiceMatrix}
			0 & 1 \\
			1 & 0
		\end{pNiceMatrix}.
	\end{equation*}

	\startsolution

	Resolvemos el problema de eigenvalores \((\op{X} - \lambda\op{I}) = 0\). Primero obtenemos los eigenvalores resolviendo \(\det(\op{X} - \lambda\op{I}) = 0\), tal que:

	\begin{align*}
		\det(\op{X} - \lambda\op{I}) & = \begin{vNiceMatrix}
			                                 -\lambda & 1        \\
			                                 1        & -\lambda
		                                 \end{vNiceMatrix}, \\
		\implies \lambda^{2} - 1     & = 0,                  \\
		\implies \lambda^{2}         & = 1.
	\end{align*}

	Por lo tanto, los eigenvalores son \(\lambda_{1} = 1\) y \(\lambda_{2} = -1\).

	Ahora, obtenemos los eigenvectores. Para \(\lambda_{1} = 1\):

	\begin{align*}
		\begin{pmatrix}
			-1 & 1  \\
			1  & -1
		\end{pmatrix} \begin{pmatrix}
			              x_{1} \\
			              x_{2}
		              \end{pmatrix} & = 0,                                \\
		\implies x_{2}                & = x_{1}\; \wedge\; x_{1} = x_{1}.
	\end{align*}

	Por lo tanto, el eigenvector normalizado correspondiente a \(\lambda_{1} = 1\) es:

	\begin{empheq}[box=\resultbox]{equation*}
		\vect{x}_{1} = \dfrac{1}{\sqrt{2}}\begin{pmatrix}
			1 \\
			1
		\end{pmatrix}.
	\end{empheq}

	Ahora, para \(\lambda_{2} = -1\):

	\begin{align*}
		\begin{pmatrix}
			1 & 1 \\
			1 & 1
		\end{pmatrix} \begin{pmatrix}
			              x_{1} \\
			              x_{2}
		              \end{pmatrix} & = 0,                                 \\
		\implies x_{2}                & = -x_{1}\; \wedge\; x_{1} = x_{1}.
	\end{align*}

	Por lo tanto, el eigenvector normalizado correspondiente a \(\lambda_{2} = -1\) es:

	\begin{empheq}[box=\resultbox]{equation*}
		\vect{x}_{2} = \dfrac{1}{\sqrt{2}}\begin{pmatrix}
			1 \\
			-1
		\end{pmatrix}.
	\end{empheq}
\end{problema}
\end{document}
